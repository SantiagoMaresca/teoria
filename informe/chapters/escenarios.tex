\section{Compradores reales}

Para simular la interacción con usuarios reales se genera un archivo de pruebas que en tiempos aleatorios
presenta un nuevo pedido por parte de un cliente.
La cantidad de entradas es también generada de forma aleatoria, entre 1 y 10.
El tipo de usuarios en este caso es \(t=2\), y no hay asientos reservados para clientes con discapacidades.

\subsubsection{Tiempo de respuesta}
El tiempo entre la solicitud de una entrada y la respuesta correspondiente se conoce como el \emph{tiempo de respuesta}.
Para cada solicitud realizada por los clientes se medirá el tiempo de respuesta.
Finalizada la simulación se calculará el promedio de este.

Además del promedio se medirá la relación entre \(t_i\) y \(n_i\), donde \(t_i\) es el tiempo de respuesta para el i-ésimo
cliente y \(n_i\) es la cantidad de entradas que solicita.

Se distinguirá para este último caso entre las ventas con éxito (considerando solo las interacciones que finalicen en concretar la venta)
y las demás interacciones (que finalizan con un error por \emph{entrada no disponible}).

\section{Compradores concurrentes aleatorios} \label{escenario-tiempoespera}
Considerando un único espectáculo, se introducen clientes al sistema.
Cada uno de ellos tiene una cantidad de entradas que desea comprar
(sea \(n_i\)), sin importar su ubicación.

Estos intentan realizar la compra de asientos cualesquiera en el espectáculo.
En el caso de que la compra no sea exitosa (otro cliente adquirió primero) se procederá a intentarlo otra vez.
Se introducirán en total \(m\) clientes, donde \(N = \sum_{i=1}^{m} n_i\)
es la cantidad total de entradas disponibles para el evento\footnote{
	Para cada cliente se calcula un número aleatorio con distribución uniforme.
	De esta forma 
	\(n_i \in [1, 8] \wedge m \in [\frac{N}{8}, N]\).
}.
De esta forma aseguramos que cada cliente logra finalizar su interacción habiendo adquirido las entradas que deseaba.

\begin{figure}
\end{figure}
\begin{figure}
\centering{
	\def\svgwidth{\linewidth}
\input{img/escenario1.pdf_tex}
\caption{\nameref{escenario-tiempoespera}}
\label{fig:escenario1}
}
\end{figure}

En esta interacción los clientes ingresan de forma concurrente al sistema, y su orden particular dependerá de la carga
particular del sistema en que ejecute la planificación.

\subsubsection{Tiempo de respuesta}
Una vez más se mide el tiempo de respuesta de los usuarios, realizando también un análisis de la relación entre el tiempo
de respuesta y la cantidad de entradas solicitada.

La hipótesis es que las ventas de mayor valor tendrán un tiempo de respuesta relativamente menor.
A pesar de su mayor tiempo de procesamiento serán atendidas previamente.


%%%%%%%%%%%%%%%%%%%%%%%%%%%%%%%%%%%%%%%%%%%%%%%%%%%%%%%%%%%%%%%%%%%%%%%%%%%%%%%%
\section{Carga del sistema}
Para un evento con \(n\) entradas en total se introducen \(m\) clientes al sistema, donde \(n < m\).
Al igual que en la sección \ref{escenario-tiempoespera}, los clientes intentarán
la compra de un conjunto aleatorio de entradas, re-intentando en caso de una falla.

\subsubsection{Tiempo de respuesta}
Al igual que en la sección \ref{escenario-tiempoespera}, medimos el tiempo de respuesta medio de las solicitudes,
y su relación para cada cliente con la cantidad de entradas.
