\section{Recursos}

\subsubsection{Asientos}

Cada asiento tiene una ubicación específica y única en el estadio.

\subsubsection{Lugares en cancha}

El tamaño de la cancha determina la cantidad de lugares, pero estos no tienen
cada uno una ubicación específica como los asientos.

\subsubsection{Esquema del estadio}

El esquema es la distribución física de los espacios (asientos y cancha) en él.

\subsubsection{Entradas}

Las entradas, asignadas a un espacio y un espectáculo en particular, poseen
además un precio determinado.

\subsubsection{Espectáculos}\label{recurso:espectaculo}

Los espectáculos tienen una fecha y hora, duración, nombre, así como entradas asociadas.

\subsubsection{Agencias, Organizadores de eventos y puntos de venta}

Estas entidades poseen la capacidad de funcionar como intermediarios en la venta
de entradas a usuarios finales. Deben ser capaces de identificarse ante la
plataforma como tales, para asegurar permisos diferentes.

En varios casos, según se dictamine al agregar el espectáculo a la plataforma,
estos podrían poseer preferencia o derecho sobre las entradas o un subconjunto
de estas.

\section{Procesos}

\subsubsection{Agregar un Espectáculo}

Al agregar un espectáculo al sistema debe darse información sobre el mismo, como
su nombre, categoría y duración.  Además, deben especificarse categorías y
precios de las entradas, según localidad.

\subsubsection{Obtener entradas disponibles}

Debe ser posible obtener un listado de los asientos o entradas disponibles para
un evento determinado.

\subsubsection{Identificación/Validación de usuarios}

Los usuarios del sistema (agencias, puntos de venta y usuarios de la aplicación o sitio web)
deben ser capaces de validar su identidad para asegurar un
funcionamiento seguro del mismo.

\subsubsection{Registro en la plataforma}

Los usuarios del sitio web o la aplicación deben ser capaces de registrarse
para poder luego realizar compras.

\subsubsection{Venta de entradas a privados}

Un usuario debe ser capaz de realizar la compra de una o más
entradas para un espectáculo determinado. Pueden aplicarse restricciones
(e.g.\ asientos para público con capacidades diferentes o asientos no-contiguos).

\subsubsection{Venta de entradas a mayoristas}

Como tales, y eventualmente solo si el espectáculo en particular lo permitiera, los mayoristas\footnote{Agencias u organizadores} serán
capaces de realizar la compra de un lote (cantidad definida por espectáculo) de entradas,
a un precio por entrada menor que en la venta a privados.

\subsubsection{Correo de confirmación al usuario al comprar entrada}

Cada cliente que reserva un asiento debe recibir un correo que confirme la compra que acaba de realizar. 
El correo debe contener la información del evento y la información de la entrada.
Esta sección no se implementará hasta la entrega del producto final con el fin de simplificar las simplificar
las simulaciones de distintos escenarios.