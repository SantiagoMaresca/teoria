\section{\nameref{escenario-tiempoespera}}

Como primera, y obvia, conclusión, se observa que, efectivamente, se realiza una venta de \(N\) entradas, y no más ni menos,
por lo que no se observan errores de lógica.

En el caso de la simulación ejecutada, tenemos \(N = 10000\), y la cantidad de clientes que compran cada número de entradas
se presenta en la tabla \ref{table:3}.
\begin{table}[H]
\centering
\begin{tabular}{ccc}
	N\textsuperscript{o} de entradas & N\textsuperscript{o} de clientes & Tiempo medio de respuesta (ms)  \\
\hline
1 &  255 & 16.434088\\
2 &  295 & 18.517253 \\
3 &  261 & 13.974438\\
4 &  271 & 17.137831\\
5 &  270 & 15.220248\\
6 &  263 & 17.686285\\
7 &  296 & 16.506400 \\
8 &  286 & 16.822844\\
\end{tabular}
\caption{Desglose por cantidad de entradas}
\label{table:3}
\end{table}


Los tiempos de respuesta fueron medidos en cada interacción para luego calcular su promedio, según algunos criterios.
En primer caso se analiza el tiempo de respuesta promedio para toda transacción,
incluyendo las comunicaciones con el sistema de tipo ``consulta de entradas'', así como las ventas fallidas.
Por otro lado se analiza el tiempo de respuesta de las transacciones que acaban con una venta concretada.
Los resultados se presentan en la tabla \ref{table:2}.
Además se presenta en la tabla \ref{table:3} el tiempo medio para cada cantidad de entradas.

\begin{table}[H]
\centering
\begin{tabular}{l|r}
	{} &  Tiempo de respuesta medio (ms) \\
	\hline
Normal & 2.747720 \\
Transacciones exitosas  & 25.564688 \\
\end{tabular}
\caption{Tiempo de respuesta medio}
\label{table:2}
\end{table}

De estos datos se concluye que el tiempo medio de respuesta es independiente de la cantidad de entradas adquiridas.
No es posible concluir que a mayor cantidad de entradas el tiempo de respuesta disminuya, pero observamos que el tiempo tampoco es proporcional a la cantidad de entradas.
Es probable que el mayor tiempo de procesamiento sea contrarrestado por el hecho de darle atención temprana a las compras de mayor valor económico.

Estos resultados se presentan en la figura \ref{fig:respuesta_concurrente}, mostrando en rojo el valor medio.

\begin{figure}[H]
\centering{
        \includegraphics[width=0.7\textwidth]{grafica_1}
\caption{Tiempo de respuesta vs. cantidad de entradas}
}
\label{fig:respuesta_concurrente}
\end{figure}

\section{Carga del sistema}

En el caso de la simulación ejecutada, tenemos \(N = 10000\), y la misma cantidad de clientes. La cantidad de clientes que compran según
las distintas cantidades de entradas que se presentan en la tabla \ref{table:carga}.

\begin{table}[H]
\centering
\begin{tabular}{ccc}
	N\textsuperscript{o} de entradas & N\textsuperscript{o} de clientes & Tiempo medio de respuesta (ms)  \\
\hline
1                 & 1206 &            24.575421 \\
2                 & 1259 &            21.579129 \\
3                 & 1253 &            26.233702 \\
4                 & 1266 &            24.669094 \\
5                 & 1260 &            25.211271 \\
6                 & 1274 &            27.186334 \\
7                 & 1254 &            23.700102 \\
8                 & 1228 &            25.710476 \\
\end{tabular}
\caption{Desglose por cantidad de entradas}
\label{table:carga}
\end{table}

Una vez más observamos que el tiempo de respuesta es independiente de la cantidad de entradas solicitadas.
Los resultados se presentan en la tabla \ref{table:carga2}.
Además se presenta en la tabla \ref{table:carga} el tiempo medio para cada cantidad de entradas.
\begin{table}[H]
\centering
\begin{tabular}{l|r}
	{} &  Tiempo de respuesta medio (ms) \\
	\hline
Normal & 7.101191 \\
Transacciones exitosas  & 24.801971 \\
\end{tabular}
\caption{Tiempo de respuesta medio}
\label{table:carga2}
\end{table}

Al analizar los tiempos de respuesta se puede ver que en comparación con la simulación previa, estos no varían al aumentar significativamente la cantidad de clientes.
Estos resultados se presentan en la figura \ref{fig:respuesta_carga}, mostrando en rojo el valor medio.

\begin{figure}[H]
\centering{
        \includegraphics[width=0.7\textwidth]{graficaMuchos1}
\caption{Tiempo de respuesta vs. cantidad de entradas}
}
\label{fig:respuesta_carga}
\end{figure}

\section{Compradores reales}

En el caso de la simulación ejecutada, tenemos \(N = 5000\), y \(M = 10000\) clientes. La cantidad de clientes que compran según
las distintas cantidades de entradas, y el tiempo medio para cada cantidad de entradas, se presentan en la tabla \ref{table:carga}.
Los compradores que realizan consultas de qué entradas se encuentran disponibles se muestran al final de la misma.

\begin{table}[H]
\centering
\begin{tabular}{ccc}
	N\textsuperscript{o} de entradas & N\textsuperscript{o} de clientes & Tiempo medio de respuesta (ms)  \\
\hline
1                 & 661 &            0.396308 \\
2                 & 640 &            0.824615 \\
3                 & 618 &            2.713671 \\
4                 & 685 &            -- \\
5                 & 640 &            0.364268 \\
6                 & 690 &            0.556214 \\
7                 & 624 &            -- \\
8                 & 730 &            -- \\
9                 & 649 &            2.453426 \\
10                & 684 &            -- \\
Consulta                 & 3380 &           1.338888 \\
\end{tabular}
\caption{Desglose por cantidad de entradas}
\label{table:reales}
\end{table}

Se dio en esta simulación que ninguno de los clientes que intentó comprar 4, 7, 8 o 10 entradas tuvo éxito en su reserva,
por lo tanto no se pudo registrar un tiempo de respuesta para estos clientes.
Se puede observar que no hay una dependencia entre el tiempo de respuesta y la cantidad de entradas solicitadas, lo cual
nos lleva a concluir que son efectivamente variables independientes, y que las fluctuaciones en el tiempo de respuesta se deben
al azar en los tiempos de llegada de cada cliente y la cantidad de clientes concurrentes en cada momento.

Los resultados se presentan en la tabla \ref{table:carga2} y en la figura, mostrando en rojo el valor medio \ref{fig:respuesta_reales},

\begin{table}[H]
\centering
\begin{tabular}{l|r}
	{} &  Tiempo de respuesta medio (ms) \\
	\hline
Normal & 0.649355 \\
Transacciones exitosas  & 1.331675 \\
\end{tabular}
\caption{Tiempo de respuesta medio}
\label{table:reales2}
\end{table}

\begin{figure}[H]
\centering{
        \includegraphics[width=0.7\textwidth]{grafica_reales}
\caption{Tiempo de respuesta vs. cantidad de entradas}
}
\label{fig:respuesta_reales}
\end{figure}

Una observación interesante a partir de la gráfica es que a pesar de que el tiempo de respuesta medio para las solicitudes de consulta es bajo
y similar a los tiempos de las de reserva, muchas consultas tuvieron un tiempo de respuesta mucho mayor. Esto refleja cómo el sistema
le da mayor prioridad a las solicitudes de reserva (ya que son las que generan ingreso) que a las de consulta, y por lo tanto demora las respuestas
de consulta con tal de responder rápido a las reservas.

\section{Conclusiones generales}

El uso de la estructuras de datos concurrente \type{ConcurrentHashMap} proporcionó una abstracción que permitió el desarrollo eficaz
sin mayor preocupación por el control del acceso a la estructura compartida.
Además, a diferencia de un cerrojo mutex para proteger una estructura común, nuestra estructura permite el acceso concurrente en la mayor parte de los casos,
aspecto que se observa en los tiempos de espera medidos.
La eficiencia de dar uso a la ejecución en paralelo mejoró el desempeño del sistema significativamente.

Para la instalación final del producto deberán ser implementada algunas características menores que se obviaron para presentar las simulaciones más relevantes.
En primer lugar los datos del programa (eventos, esquema de distribución de asientos del estadio, descuentos y proporción de hilos de venta contra solicitud) serán leídos
de datos locales, permitiendo que sean sobrescritos fácilmente.
Además se implementará el sistema de autenticación de usuarios, así como la opción de pago online mediante uso de tarjetas o puestos de venta.
Con un usuario autenticado será posible además poseer datos del mismo, como una dirección de correo mediante la cual comunicar sobre sus compras,
enviándole entradas digitales.
