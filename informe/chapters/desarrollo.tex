
\section{Armado de un Árbol-KDR}
\begin{lstlisting}[language=Python]
	
def makeKDRTree(points, r, dim = 0):
	if not points:
		return None # Empty trees are None
	elif len(points) == 1:
		return tuple(points) # Leaf nodes have one point.

	# Obtenemos las r medianas, para las dimensiones dim, dim+1, ..., dim+(r-1)
	medians = getMedians(points, r, dim)

	partitions = [None] * (2**r + 1) # Tendremos 2**r particiones (hijos)
	partitions[0] = medians
	i = 1
	for criteria in it.product([False, True], repeat = r):  # 2^r iteraciones
		partition = [p for p in points if matches_criteria(criteria, medians, p, dim)] # n iteraciones
		partitions[i] = makeKDRTree(partition, r, (dim+r) % len(points))
		i += 1

	return tuple(partitions)
	\end{lstlisting}
\textbf{ Orden de tiempo de ejecución del algoritmo : }
{Mejor de los casos: points es nulo o es una tupla de largo uno : 
el algoritmo tendrá orden de ejecución : O(1).
Caso promedio : 



}

	\section{Búsqueda de un Árbol-KDR}
	\begin{lstlisting}[language=Python]
	def searchKDRTree(kdrTree, r, point, dim = 0):
		if (kdrTree is None):
			return False
		elif len(kdrTree) == 1:
			return kdrTree[0] == point
		medians = kdrTree[0]
		i = 1
		for criteria in it.product([False, True], repeat = r):  # 2^r iteraciones
			if (matches_criteria(criteria, medians, point, dim)):
				# go to this subtree
				chosen_subtree = kdrTree[i]
				return searchKDRTree(chosen_subtree, r, point, dim+r)
			i += 1
	
		return False
		\end{lstlisting}
	\textbf{ Orden de tiempo de ejecución del algoritmo : }
		{Mejor de los casos: points es nulo o es una tupla de largo uno : 
		el algoritmo tendrá orden de ejecución : O(1).
		Caso promedio : 
		
		
		
		}


% \begin{algorithm}
% \caption{Busca una tupla de largo R en un arbol KDR}
% \label{alg:busqueda-kdr}
% \begin{algorithmic}[1]
% 	\Require El árbol kdr debe estar creado
% \Procedure{searchKDRTree}{kdrTree, r, point, dim = 0}
% \If {kdTree == nulo}
% \State {devolver Falso}
% \Else{}
% \If {kdTree == nulo}
% \State devolver kdrTree[0] == point

% medians =kdrTree[0]
% \ForAll{(criteria,) $\in$ productoCartesiano(false,true,r)}
% 	\State s.put(``\{i\}) \{evento.getNombre()\} \{evento.getCuando()\}'');
% \EndFor
% \EndProcedure
% \end{algorithmic}
% \end{algorithm}

% El algoritmo \ref{alg:crear-kdr} muestra la respuesta del sistema cuando se detalla un id de evento.
% En este caso se representa, para el evento determinado, qué entras están disponibles (mediante id),
%     así como su precio.