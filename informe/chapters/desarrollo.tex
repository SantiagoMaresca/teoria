\section{Creación de un Árbol-KDR}

\begin{algorithm}
\caption{Crea un arbol KDR}
\label{alg:crear-kdr}
\begin{algorithmic}[1]
	\Require \emph{eventos} es un vector de \type{Event}
\Procedure{listarEventos}{Socket s}
\State s.put(``Los eventos disponibles son'');
\ForAll{(i, evento) $\in$ eventos}
	\State s.put(``\{i\}) \{evento.getNombre()\} \{evento.getCuando()\}'');
\EndFor
\EndProcedure
\end{algorithmic}
\end{algorithm}

El algoritmo \ref{alg:crear-kdr} muestra la respuesta del sistema cuando se detalla un id de evento.
En este caso se representa, para el evento determinado, qué entras están disponibles (mediante id),
	así como su precio.



	\begin{algorithm}
		\caption{Listar las entradas disponibles}
		\label{alg:listar-entradas}
		\begin{algorithmic}[1]
		\Procedure{listar}{Socket s, Tokens solicitud}
			\State s.put(``Las entradas disponibles son: '');
			\State idEvent $\gets$ solicitud.nextToken()
			\State evento $\gets$ events[idEvento]
			\ForAll{(entrada, costo) $\in$ evento.soldTickets.entradasDisponibles()}
				\State s.put(``\{entrada\}) \$\{costo\}'');
			\EndFor
		\EndProcedure
		\Statex
		\Require \emph{st.validIds} es un mapa de id entrada a costo (vendidas o no)
		\Require \emph{st.asientosOcupados} es el mapa descrito en \ref{concurrenthm}
		\Procedure{entradasDisponibles}{SoldTickets st}
			\ForAll{(idEntrada, costo) $\in$ st.validIds}
			\If{st.asientosOcupados.get(entrada) $=$ \textbf{null}}
				\State \textbf{yield} (idEntrada, costo)
			\EndIf
			\EndFor
		\EndProcedure
		\end{algorithmic}
		\end{algorithm}